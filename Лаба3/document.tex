\documentclass[]{article}
\usepackage[T2A]{fontenc}
\usepackage[utf8]{inputenc}
\usepackage[russian]{babel}
\usepackage{amsmath}
\usepackage{amsmath, amsfonts, amssymb, amsthm, mathtools}
\usepackage[left=20mm, top=15mm, right=15mm, bottom=20mm, nohead, nofoot]{geometry}
\usepackage{graphicx}
\usepackage{float}%"Плавающие" картинки
\usepackage{wrapfig}%Обтекание фигур (таблиц, картинок и прочего)
\usepackage{listings}
%opening
\begin{document}
	\begin{titlepage}
		\begin{center}
			\large Санкт-Петербургский политехнический университет Петра Великого \\
			\large Физико-механический институт \\
			\large Высшая школа теоретической механики и математической физики \\[2cm] % [] - отступ
			\large Направление подготовки \\
			\large 01.03.03 Механика и математическое моделирование \\[2cm]
			\LARGE \textbf {Отчёт по лабораторной работе №2} \\[0.5cm]
			\LARGE \textbf {Тема: "Динамический гаситель"} \\[0.5cm]
			\large Дисциплина "Теория колебаний" \\[4cm]
		\end{center}
		\begin{minipage}{0.25\textwidth} % врезка в половину ширины текста
			\begin{flushright}
				\large\textbf{Выполнили:}\\
				\large Работинский А.Д. \\
				\large Зеленкина Е.Е. \\
				\large Дёмин М.Д\\
				\large Исхакова Э.Р. \\
				\large {Группа:} 5030103/10001 \\
				\large \textbf{Преподаватель:}\\
				\large О.С. Лобода
			\end{flushright}
		\end{minipage}
		\mbox{}
		\vfill
		\begin{center}
			\large Санкт-Петербург \\
			\large 2022 
		\end{center} 
	\end{titlepage}
	
	\newpage
	\section*{1) Постановка задачи}
	Рассмотрим систему, состояющую из двух грузов и пружины массами и жесктостями соответственно $m_0, m_1, k_0,k_1$
	
\end{document}
